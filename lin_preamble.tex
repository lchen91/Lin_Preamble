\usepackage{paralist}
\usepackage{amsmath,amssymb,dsfont,amsthm,dsfont,hyperref}
\usepackage{xfrac}
\usepackage{xcolor}
% \usepackage[title,toc,titletoc,page]{appendix}
%\usepackage[linesnumbered,ruled]{algorithm2e}
\usepackage[numbers,sort,square]{natbib}
\usepackage[capitalize]{cleveref}
\usepackage{subcaption}
\usepackage{mathtools}
\usepackage{booktabs} % For prettier tables
\usepackage{mathrsfs}
\usepackage{enumitem}

\usepackage{forloop}

\usepackage{thmtools} 
\usepackage{thm-restate}
\declaretheorem[name=Theorem,numberwithin=section]{thm}

\newtheorem{theorem}{Theorem}
\newtheorem{lemma}[theorem]{Lemma}
\newtheorem{cor}[theorem]{Corollary}
\newtheorem{obs}[theorem]{Observation}
\newtheorem{prop}[theorem]{Proposition}

% \crefname{appsec}{Appendix}{Appendices}


\theoremstyle{definition}
\newtheorem{definition}{Definition}
\newtheorem{remark}{Remark}

% Caligraphics and Board Letters
\newcommand{\defcal}[1]{\expandafter\newcommand\csname 
	c#1\endcsname{{\mathcal{#1}}}}
\newcommand{\defbb}[1]{\expandafter\newcommand\csname 
	b#1\endcsname{{\mathbb{#1}}}}
\newcommand{\defbf}[1]{\expandafter\newcommand\csname 
	bf#1\endcsname{{\mathbf{#1}}}}
\newcounter{calBbCounter}
\forLoop{1}{26}{calBbCounter}{
	\edef\letter{\Alph{calBbCounter}}
	\expandafter\defcal\letter
	\expandafter\defbb\letter
	\expandafter\defbf\letter
}
\forLoop{1}{26}{calBbCounter}{
	\edef\letter{\alph{calBbCounter}}
	% 	\expandafter\defcal\letter
	% 	\expandafter\defbb\letter
	\expandafter\defbf\letter
}

% Other Commands
\newcommand{\eps}{\varepsilon}
\newcommand{\E}{\mathbb{E}}
\newcommand{\ind}{\mathds{1}}
\DeclareMathOperator{\sign}{sign}
\DeclareMathOperator{\diag}{diag}

% \newcommand{\ie}{{\it i.e.}}
% \newcommand{\eg}{{\it e.g.}}
\newcommand{\lc}[1]{\textcolor{red}{\textbf{\colorbox{yellow}{Lin:} }#1}}
% \newcommand{\lc}[1]{}

\DeclareMathOperator{\unif}{Unif}
\DeclareMathOperator{\pois}{Poisson}
\DeclareMathOperator{\ber}{Ber}
\DeclareMathOperator{\bin}{Bin}
\DeclareMathOperator{\vol}{vol}
\DeclareMathOperator*{\argmax}{arg\,max}
\DeclareMathOperator*{\argmin}{arg\,min}
\DeclareMathOperator{\im}{im}
\DeclareMathOperator{\tr}{tr}
\DeclareMathOperator{\rank}{rank}
